\documentclass{article}
\usepackage{booktabs}
\usepackage{multirow}
\usepackage{amsmath}
\usepackage{gensymb}
\usepackage{graphicx}
\usepackage{bm}
\usepackage[spanish]{babel}
\def\spanishtablename{Tabla}
\usepackage[a4paper,margin=1in]{geometry}
\usepackage[utf8]{inputenc}
\usepackage{adjustbox}
\usepackage{amsmath}
\usepackage{float}
\usepackage{tikz}
\usetikzlibrary{calc}
\usetikzlibrary{arrows}
\usepackage{pgfplots}
\usepackage{subcaption} 



\title{Modificación enzimática de sémola proveniente de \textit{Triticum durum}}
\author{Departamento de Bioingeniería, Tecnológico de Monterrey Campus Toluca.\\ Eduardo Monroy Cárdenas 2000, San Antonio Buenavista 50110, Toluca, Estado de México.}
\date{\today}
\begin{document}


\maketitle


\section{Introducción}
	
La sémola procedente de \textit{Triticum durum} es el ingrediente principal en la fabricación de productos alimenticios como la pasta. Dicha sémola está compuesta por proteínas, grasas, carbohidratos y vitaminas, siendo el almidón la macromolécula más abundante, abarcando entre 60\% y 65\% de la composición total de la sémola [1]. Este último porcentaje presenta un problema para los diabéticos, ya que cuando el almidón es ingerido, la enzima $\alpha$-amilasa, que se encuentran en el tracto digestivo, comienza a degradarlo liberando moléculas de glucosa que hacen de la sémola y posteriormente a la pasta como un producto de alto índice glicémico. [2]\\

El almidón está formado a partir de dos polisacáridos; amilosa y amilopectina. La amilosa está formada por $\alpha$-D glucopiranosas unidas mediante enlaces $\alpha$1-4 que se acomoda de manera helicoidal conformando entre el 25 y 30\% del almidón. La amilopectina representa el 70 o 75\% restante de la macromolécula, forma una cadena ramificada de $\alpha$-D glucopiranosas unidas por enlaces $\alpha$ 1-6 [3]. Se ha encontrado que un almidón con altas proporciones de amilosa es de difícil degradación para $\alpha$-amilasa[4].\\

Existen enzimas que nos permiten modificar esta macromolécula para generar una más resistente a la degradación enzimática de $\alpha$-amilasa. Se ha encontrado que tanto  Pululanasa (EC 3.2.1.41) e Isoamilasa (EC 3.2.1.68) generan la hidrólisis en los enlaces $\alpha$ 1-6 glucosídicos del almidón, posteriormente, dichas enzimas permiten la formación de nuevas cadenas de amilosa, reduciendo el índice glicémico del almidón. [5]\\

Para medir la proporción de amilosa en el almidón, se ha encontrado que el yodo puede formar un complejo con amilosa, el cual puede ser detectado en un espectrofotómetro a una longitud de onda de aproximadamente 610 nm [6]. Por otra parte, la cantidad de azúcares reductores como la glucosa, puede ser medido utilizando el reactivo de DNS y un espectrofotómetro a una longitud de onda de aproximadamente 550 nm [7].\\

\section{Objetivos}
\begin{itemize}
	\item Caracterización  de $\alpha$-amilasa, pululanasa e isoamilasa
	\item Reducción del índice glicémico de la sémola de \textit{Triticum durum} a través de modificación enzimática.	
\end{itemize}

\section{Metodología}

Es importante destacar que para todas la muestras realizadas en tubos de ensaye, para el tiempo 0 se aplicó primero el DNS o yodo (impiden reacción enzimática) y después el volumen correspondiente de enzima. Para muestras con matraz, antes de añadir la enzima, se obtuvieron las muestras de tiempo  0 con un volumen de 0.425 mL, en donde después de añadío 0.075 mL de buffer correspondiente.

\subsection{ Experimentos con \bm{$\alpha$}-amilasa}

\subsubsection{Efecto de temperatura en \bm{$\alpha$}-amilasa}
Para medir la actividad de $\alpha$-amilasa, se tomaron 11 muestras por triplicado, formando un total de 33 tubos de ensaye utilizados. Se colocaron 0.250 mL de almidón al 1.5\% (m/v), 0.2 mL de buffer de acetatos 100 mM a pH 5 a todos los tubos y se atemperaron en un termobaño a la temperatura de reacción correspondiente por 5 minutos, posteriormente se añadieron 0.050 mL de solución de enzima  para obtener un volumen total de reacción de 0.5 mL y se dejó correr el ensayo dentro del termobaño. La reacción se detuvo cada dos minutos añadiendo 1 mL de DNS a los tres tubos de cada serie hasta completar los 20 minutos de reacción. Posteriormente los tubos se pusieron en agua a 95 ºC por 5 minutos y se enfriaron en chorro de agua fría, se colocaron 2.5 mL de agua destilada y se leyeron en el espectrofotómetro a 550 nm.\\

La concentración inicial de las solución stock de enzima fue de 3.985 U/mL diluida en dH20 y la reacción se llevó a cabo a 55ºC. Más adelante, se cambió la concentración de enzima a 7.97 U/mL diluida en buffer de acetatos 100 mM a pH 5. Utilizando esta última configuración, se variaron  las temperaturas de reacción a 30, 35, 40, 45, 50, 55 y  60ºC.

\subsubsection{Efecto de \bm{$\alpha$}-amilasa en amilosa perteneciente al almidón encontrado en la sémola de trigo}
	
En un matraz de 25 mL se colocaron 15 mL de sémola al 0.5\% (m/v), 12 mL de buffer de acetatos 100 mM a pH 5 con 1\% de albúmina sérica bovina, se atempero por 10 minutos a 50ºC en incubadora. Posteriormente, al cerciorarse que se atemperó correctamente, se añadieron 3 mL de solución de $\alpha$-amilasa con una concentración de 15.94 U/mL y se dejó reaccionar por 60 minutos en incubadora a 120 rpm de agitación, tomando 0.5 mL de muestra cada 10 minutos por triplicado. Para revelar la muestras se les añadió 0.5 mL de Yodo-Lugol a 0.01M y 11.5 mL de dH2O. Posteriormente se centrifugaron las muestras a 40ºC, 4000 rpm por 10 minutos. Finalmente,  2 mL de la cada muestra se filtraron usando un filtro pirinola de 45 micrómetros, el filtrado se leyó usando el espectrofotómetro a 610 nm. Se realizó un control, en el cual se utilizó la misma metodología descrita anteriormente, únicamente en vez de agregar $\alpha$-amilasa se sustituyó el mismo volumen por buffer de acetatos 100 mM a pH 5.

\subsubsection{Prueba de índice glicémico en sémola de trigo y almidón puro}
En dos matraces (A y B) de 100 mL, se agregaron 16.5 mL de una solución de sémola de trigo al 0.5\% (m/v) en el matraz A y 16.5 mL de una solución de almidón puro al 0.5\% (m/v) en el matraz B. En ambos matraces se añadieron 13.35 mL de buffer de acetatos 100 mM a un pH de 5. Se dejaron atemperar ambos matraces a 50ºC durante 10 minutos, cuando se cercioró que los matraces se encontraban a 50ºC se añadió en ambos una solución de $\alpha$-amilasa a 15.94 U/mL y se dejó reaccionar por 300 minutos. Durante los primeros 30 minutos se obtuvieron 0.5 mL de muestra por triplicado cada 5 minutos. Los 270 minutos restantes se obtuvieron 0.5 mL de muestra cada 30 minutos. Posteriormente los tubos se pusieron en agua a 95ºC por 5 minutos y se enfriaron en chorro de agua fría, se colocaron 2.5 mL de agua destilada y se leyeron en el espectrofotómetro a 550 nm.

\subsection{ Experimentos con pululanasa}

\subsubsection{Efecto de temperatura y pH en Pululanasa}
	Para medir actividad de pululanasa, se tomaron 11 muestras por triplicado para formar un total de 33 tubos de ensaye utilizados. Se colocaron 0.250 mL de almidón,  0.2 mL de buffer a cada uno los tubos y se atemperaron en un termobaño a la temperatura de reacción correspondiente por 5 minutos. Posteriormente se añadieron 0.050 mL de solución de enzima para obtener un volumen total de reacción de 0.5 mL y se dejó correr el ensayo en el termobaño. La reacción se detuvo cada 1 minuto añadiendo 1 mL de DNS a los tres tubos de cada serie hasta completar los 10 minutos de reacción. Posteriormente los tubos se pusieron en agua a 95ºC por 5 minutos y se enfriaron en chorro de agua fría, se colocaron 2.5 mL de agua destilada y se leyeron en el espectrofotómetro a 550 nm.\\
	
	Inicialmente se utilizó una concentración stock de almidón y enzima al 1.5\% (m/v) y 1 U/mL respectivamente y se realizó la reacción a una temperatura de 55ºC utilizando un buffer de citratos 100 mM a pH 6. Posteriormente, la concentración de almidón se ajustó a 1\% (m/v) y la concentración de enzima a 0.5 U/mL. Siguiendo esta última conformación, se variaron las temperaturas de reacción a 45, 50 y 55ºC utilizando buffer de citratos 100 mM a pH 6. Tambíen se varió el pH de reacción mediante un buffer de acetatos 100 mM a pH 5, buffer de citratos 100 mM a pH 6 y un buffer de fosfatos 100 mM a ph 7 en una temperatura de reacción de 50ºC.
	
\subsubsection{Efecto de pululanasa en amilosa y azúcares reductores pertenecientes al almidón encontrado en la sémola de trigo}	

En dos matraces (A y B) de 100 mL, se agregaron 24.5 mL de sémola de trigo al 0.5\% (m/v). En ambos matraces se añadieron 5.25 mL de buffer. Se dejaron atemperar ambos matraces a 50ºC durante 10 minutos, cuando se cercioró que los matraces se encontraban a 50ºC se añadió en ambos una solución de pululanasa a 0.5 U/mL y se dejó reaccionar por 60 minutos. Cada 10 minutos se obtuvieron 0.5 mL de muestra por triplicado de cada matraz. Para las muestras provenientes del matraz A, se les añadió 0.5 mL de Yodo-Lugol a 0.01M y 11.5 mL de dH2O. Posteriormente se centrifugaron las muestras a 40ºC, 4000 rpm por 10 minutos. Finalmente,  2 mL de la cada muestra se filtraron usando un filtro pirinola de 45 micrómetros, el filtrado se leyó usando el espectrofotómetro a 610 nm. Para las muestras provenientes del matraz B, añadiendo 1 mL de DNS y se pusieron en agua a 95ºC por 5 minutos, se enfriaron en chorro de agua fría, se colocaron 2.5 mL de agua destilada y se leyeron en el espectrofotómetro a 550 nm. Se varió el experimento utilizando dos buffers acetatos 100 mM a pH 3.5 y 5.  Para cada metodología descrita, se realizó un control en el cual se sustituyó el volumen de isoamilasa por el buffer correspondiente a la reacción.

\subsection{Experimentos con isoamilasa}

\subsubsection{Efecto de temperatura y pH en isoamilasa}
Para medir actividad de isoamilasa, en un matraz de 25 mL se colocaron 14.7 mL de sémola, 3.15 mL de buffer de acetatos 100 mM a pH 3.5 con 1\% de albúmina sérica bovina, se atempero por 10 minutos a la temperatura de reacción en incubadora, posteriormente, al cerciorarse que se atemperó correctamente, se añadió 2.7 mL de solución de enzima y se dejó reaccionar por 60 minutos en incubadora a 120 rpm de agitación, tomando 0.5 mL de muestra por triplicado cada 10 minutos. Para revelar la muestras se les añadió 0.5 mL de Yodo-Lugol a 0.01M y 11.5 mL de dH2O. Posteriormente se centrifugaron las muestras a 40ºC, 4000 rpm por 10 minutos. Finalmente,  2 mL de la cada muestra se filtraron usando un filtro pirinola de 45 micrómetros, el filtrado se leyó usando el espectrofotómetro a 610 nm. \\

Se utilizó una concentración inicial de sémola 0.1\% (m/v) y 250 U/mL de enzima, la reacción se llevó a cabo bajo una temperatura de 40ºC. Posteriormente, se utilizaron concentraciones de sémola y enzima al 0.5\% (m/v) y 2,500 U/mL a un mismo pH y temperatura de reacción. Utilizando estas últimas concentraciones, se variaron las temperaturas en 30, 35, 40, 50 y 60ºC. Para cada metodología descrita, se realizó un control en el cual se sustituyó el volumen de isoamilasa por el buffer correspondiente a la reacción.

\subsubsection{Inhibición de Isoamilasa por glucosa}

Utilizando dos matraces (A y B) de 25 mL se colocaron 14.7 mL de sémola de trigo a una concentración de 0.5\% (m/v). En ambos matraces, se colocaron 3.15 mL de buffer de acetatos 100 mM a pH 3.5 con 1\% de albúmina sérica bovina y glucosa con concentración de 0.1 mg/mL en el matraz A y 14.7 mL de glucosa con concentración de 0.2 mg/mL en el matraz B, se atemperaron por 10 minutos a la 40ºC en incubadora, posteriormente, al cerciorarse que se atemperaron correctamente, se añadió en ambos matraces 2.7 mL de solución de isoamilasa a 25 U/mL y se dejaron reaccionar por 60 minutos en incubadora a 120 rpm de agitación. Se tomaron 0.5 mL de muestra cada 10 minutos en cada matraz por sextuplicadp. Para revelar la muestras, a tres muestras del sectuplicado se les añadió 0.5 mL de Yodo-Lugol a 0.01M y 11.5 mL de dH2O. Posteriormente se centrifugaron las muestras a 40ºC, 4000 rpm por 10 minutos. Finalmente,  2 mL de la cada muestra se filtraron usando un filtro pirinola de 45 micrómetros, el filtrado se leyó usando el espectrofotómetro a 610 nm. A las tres muestras restantes, se añadieron 1 mL de DNS y se pusieron en agua a 95ºC por 5 minutos, se enfriaron en chorro de agua fría, se colocaron 2.5 mL de agua destilada y se leyeron en el espectrofotómetro a 550 nm.  

\subsection{Modificación enzimática de sémola de \textit{Triticum durum}}
	En un matraz de 100 mL se colocaron 46.2 mL de sémola proveniente de \textit{Triticum durum} a 0.5\%, 9.9 mL de buffer de acetatos pH 5 100 mM con 1\% de albúmina sérica bovina se atempero por 10 minutos a 50ºC y al cerciorarse que se llegó a dicha temperatura, se añadió 9 ml de pululanasa a 0.5 U/mL. Se dejó reaccionar por 40 minutos tomando 0.5 mL de muestra por triplicado para revelar con Yodo y por triplicado para revelar con DNS cada 20 minutos. Se añadieron 0.2 mL de ácido acético al 99\% para bajar el pH a 3.5, se atempero por 10 minutos a 40ºC y al cerciorarse que se llegó a la temperatura, se añadieron 6.3 mL de isoamilasa a 2500 U/mL. La reacción transcurrió durante una hora tomando 0.5 mL de muestra por triplicado para revelar Yodo y por triplicado para revelar con DNS cada 10 minutos. El volumen sobrante del matraz se centrifugó a 40ºC, 4000 rpm por 10 minutos, se decantó el sobrenadante y se resuspendió el pellet en 9 mL de agua destilada, posteriormente se agregó 1.9 mL de buffer de acetatos pH 5 100 mM con 1\% de albúmina sérica bovina, se atempero a 50ºC y se añadieron 1.47 mL de $\alpha$-amilasa a 2.55 U/ml. Esta reacción se llevó a cabo durante una hora tomando 0.5 mL de muestra por triplicado para revelar con Yodo y por triplicado para revelar con DNS cada 20 minutos.\\
	
	Cabe destacar que el método de revelar amilosa con Yodo es el mismo que se utilizó para la caracterización de isoamilasa, en donde se añaden a la muestra 0.5 mL de Yodo-Lugol a 0.01M y 11.5 mL de agua destilada respectivamente. Se centrifuga a 40ºC, 4000 rpm por 10 minutos y finalmente, 2 mL de la cada muestra se filtran usando un filtro pirinola de 45 micrómetros, el filtrado se leyó usando el espectrofotómetro a 610 nm. Para el método de revelar azúcares reductores por DNS es igual al utilizado en la caracterización de $\alpha$-amilasa y de pululanasa, en donde se añadió 1 mL de DNS, posteriormente, se pusieron las muestras en agua a 95ºC por 5 minutos y se enfriaron en chorro de agua fría, se colocaron 2.5 mL de agua destilada y se centrifugaron a las mismas condiciones que las muestras de Yodo, finalmente, se leyeron a 510 nm.\\
	
	Para el matraz control se siguió el mismo protocolo descrito anteriormente, únicamente con la sustitución del volumen correspondiente a pululanasa por buffer de acetatos 100 mM a pH 5 y el volumen correspondiente a isoamilasa por buffer de acetatos 100 mM a pH 3.5.
	
	
\section{Resultados}

Siguiendo la metodología, para las Tablas 1 a 23,  Prom. Abs  representa el promedio de absorbancias de muestras por triplicado en las cuales para revelado con DNS se leyeron a 550 nm y para revelado con   yodo a 610nm. Cabe destacar que en las mismas tablas, las siglas C.V.. \% representan coeficiente de variación  en procentaje para cada set de muestras por triplicado. El control representa las muestras sin enzima.

\subsection{ Experimentos con \bm{$\alpha$}-amilasa}

% Table generated by Excel2LaTeX from sheet 'Hoja1'
\begin{table}[H]
  \centering
  \caption{Curvas de progreso de $\alpha$-amilasa a diferentes concentraciones de enzima. }
    \begin{tabular}{cccccc}
    \toprule
    \multirow{2}[4]{*}{Tiempo (min)} & \multicolumn{2}{c}{$\alpha$-amilasa (3.985 U/mL)} &       & \multicolumn{2}{c}{$\alpha$-amilasa (7.97 U/mL)} \\
\cmidrule{2-3}\cmidrule{5-6}          & Prom. Abs.  & C.V. \% &       & Prom. Abs. & C.V. \% \\
\cmidrule{1-6}    0     & 0.373 & 1.17  &       & 0.55  & 8.25 \\
    2     & 0.499 & 6.71  &       & 0.655 & 7.87 \\
    4     & 0.49  & 0.31  &       & 0.723 & 6.31 \\
    6     & 0.564 & 3.18  &       & 0.772 & 7.70 \\
    8     & 0.598 & 4.02  &       & 0.825 & 5.33 \\
    10    & 0.589 & 0.35  &       & 0.891 & 7.84 \\
    12    & 0.683 & 0.42  &       & 0.957 & 6.00 \\
    14    & 0.692 & 0.9   &       & 0.915 & 5.47 \\
    16    & 0.684 & 1.99  &       & 1.013 & 4.29 \\
    18    & 0.702 & 6.5   &       & 1.017 & 4.89 \\
    20    & 0.737 & 1.09  &       & 0.928 & 4.10 \\
    \bottomrule
    \end{tabular}%
  \label{tab:a1}%
\end{table}%

% Table generated by Excel2LaTeX from sheet 'Hoja1'
\begin{table}[H]
	\centering
	
	\caption{Curvas de progreso de $\alpha$-amilasa a diferentes temperaturas.}
	\begin{adjustbox}{width=\textwidth}
	\begin{tabular}{cccccccccccc}
		\toprule
		\multirow{2}[4]{*}{Tiempo (min)} & \multicolumn{2}{c}{Temperatura 30ºC} &       & \multicolumn{2}{c}{Temperatura 35ºC} &       & \multicolumn{2}{c}{Temperatura 40ºC} &       & \multicolumn{2}{c}{Temperatura 45ºC} \\
		\cmidrule{2-12}          & Prom. Abs. & C.V. \% &       & Prom. Abs. & C.V. \% &       & Prom. Abs. & C.V. \% &       & Prom. Abs. & C.V. \% \\
		\midrule
		0     & 0.356 & 0.65  &       & 0.444 & 1.63  &       & 0.235 & 1.61  &       & 0.381 & 9.60 \\
		2     & 0.501 & 1.45  &       & 0.498 & 1.00  &       & 0.275 & 6.62  &       & 0.478 & 7.78 \\
		4     & 0.565 & 2.32  &       & 0.550 & 3.69  &       & 0.296 & 5.49  &       & 0.534 & 1.74 \\
		6     & 0.562 & 1.04  &       & 0.639 & 1.26  &       & 0.355 & 2.85  &       & 0.546 & 9.46 \\
		8     & 0.555 & 0.88  &       & 0.650 & 0.39  &       & 0.405 & 3.18  &       & 0.601 & 7.98 \\
		10    & 0.620 & 1.26  &       & 0.650 & 0.39  &       & 0.440 & 3.31  &       & 0.588 & 6.60 \\
		12    & 0.678 & 0.87  &       & 0.690 & 0.82  &       & 0.454 & 6.95  &       & 0.610 & 3.72 \\
		14    & 0.740 & 0.91  &       & 0.649 & 10.13 &       & 0.492 & 1.55  &       & 0.659 & 1.61 \\
		16    & 0.765 & 0.40  &       & 0.724 & 0.90  &       & 0.504 & 1.19  &       & 0.681 & 2.38 \\
		18    & 0.778 & 0.41  &       & 0.763 & 0.95  &       & 0.525 & 0.57  &       & 0.680 & 1.57 \\
		20    & 0.790 & 0.51  &       & 0.800 & 0.93  &       & 0.562 & 2.96  &       & 0.733 & 0.76 \\
		\bottomrule
	\end{tabular}%
	\end{adjustbox}
	\label{tab:a2}%
\end{table}%

% Table generated by Excel2LaTeX from sheet 'exp'
\begin{table}[H]
	\centering
	\caption{Curvas de progreso de  $\alpha$-amilasa a diferentes temperaturas.}
	\begin{tabular}{ccccccccc}
		\toprule
		\multirow{2}[4]{*}{Tiempo (min)} & \multicolumn{2}{c}{Temperatura 50ºC} &       & \multicolumn{2}{c}{Temperatura 55ºC} &       & \multicolumn{2}{c}{Temperatura 60ºC} \\
		\cmidrule{2-9}          & Prom. Abs. & C.V. \% &       & Prom. Abs. & C.V. \% &       & Prom. Abs. & C.V. \% \\
		\midrule
		0     & 0.269 & 2.89  &       & 0.249 & 16.38 &       & 0.270 & 8.38 \\
		2     & 0.307 & 0.82  &       & 0.303 & 10.33 &       & 0.297 & 4.89 \\
		4     & 0.340 & 2.21  &       & 0.327 & 12.25 &       & 0.292 & 6.75 \\
		6     & 0.393 & 0.96  &       & 0.337 & 4.16  &       & 0.345 & 4.43 \\
		8     & 0.426 & 1.82  &       & 0.354 & 13.99 &       & 0.354 & 8.21 \\
		10    & 0.444 & 4.30  &       & 0.362 & 7.84  &       & 0.374 & 0.77 \\
		12    & 0.497 & 2.46  &       & 0.414 & 11.04 &       & 0.405 & 8.57 \\
		14    & 0.578 & 1.78  &       & 0.463 & 9.11  &       & 0.413 & 6.71 \\
		16    & 0.642 & 4.36  &       & 0.473 & 2.24  &       & 0.465 & 26.35 \\
		18    & 0.679 & 0.59  &       & 0.480 & 4.02  &       & 0.428 & 10.66 \\
		20    & 0.743 & 6.53  &       & 0.514 & 5.00  &       & 0.424 & 9.41 \\
		\bottomrule
	\end{tabular}%
	\label{tab:a3}%
\end{table}%

% Table generated by Excel2LaTeX from sheet 'Hoja1'
\begin{table}[H]
	\centering
	\caption{Prueba de índice glicémico en semola de \textit{Triticum durum} y almdón puro.}
	\begin{tabular}{cccccc}
		\toprule
	
	
	
		multirow{2}[4]{*}{Tiempo (min)} & \multicolumn{2}{c}{Almidón 0.5\% (m/v)} &       & \multicolumn{2}{c}{Sémola 0.5\% (m/v)} \\
		\cmidrule{2-3}\cmidrule{5-6}          & Prom. Abs. & C.V. \% &       & Prom. Abs. & C.V. \% \\
		\midrule
		0     & 0.174 & 10.11 &       & 0.108 & 0.53 \\
		5     & 0.224 & 2.46  &       & 0.150 & 10.70 \\
		10    & 0.326 & 8.77  &       & 0.228 & 11.43 \\
		15    & 0.426 & 13.00 &       & 0.299 & 12.59 \\
		20    & 0.489 & 14.56 &       & 0.387 & 18.53 \\
		25    & 0.615 & 11.32 &       & 0.462 & 9.92 \\
		30    & 0.609 & 10.25 &       & 0.529 & 8.73 \\
		60    & 0.981 & 9.25  &       & 1.028 & 3.23 \\
		90    & 1.097 & 3.83  &       & 1.222 & 4.73 \\
		120   & 1.038 & 2.14  &       & 1.156 & 23.94 \\
		150   & 0.797 & 7.98  &       & 1.544 & 2.71 \\
		180   & 0.711 & 9.23  &       & 1.485 & 8.54 \\
		210   & 0.794 & 7.00  &       & 1.709 & 3.82 \\
		240   & 0.888 & 5.44  &       & 1.701 & 6.62 \\
		270   & 0.908 & 3.86  &       & 1.665 & 4.46 \\
		300   & 1.025 & 1.76  &       & 1.675 & 0.90 \\
		\bottomrule
	\end{tabular}%
	\label{tab:a4}%
\end{table}%

% Table generated by Excel2LaTeX from sheet 'Hoja1'
\begin{table}[H]
	\centering
	\caption{Curva de progreo de $\alpha$-amilasa utilizando yodo para revelar las muestras.}
	\begin{tabular}{cccccc}
		\toprule
		\multirow{2}[4]{*}{Tiempo (min)} & \multicolumn{2}{c}{ $\alpha$-amilasa (15.94 U/mL)} &       & \multicolumn{2}{c}{Control} \\
		\cmidrule{2-3}\cmidrule{5-6}          & Prom. Abs. & C.V. \% &       & Prom. Abs. & C.V. \% \\
		\midrule
		0     & 0.451 & 3.00  &       & 0.364 & 14.21 \\
		10    & 0.199 & 2.80  &       & 0.409 & 4.24 \\
		20    & 0.182 & 2.20  &       & 0.418 & 4.56 \\
		30    & 0.183 & 1.45  &       & 0.479 & 4.74 \\
		40    & 0.193 & 2.45  &       & 0.440 & 1.84 \\
		50    & 0.201 & 2.77  &       & 0.459 & 4.60 \\
		60    & 0.237 & 6.55  &       & 0.384 & 0.00 \\
		\bottomrule
	\end{tabular}%
	\label{tab:a5}%
\end{table}%


\subsection{ Experimentos con pululanasa}

% Table generated by Excel2LaTeX from sheet 'Hoja1'
\begin{table}[H]
	\centering
	\caption{Curvas de progreso de pululanasa a diferentes concentraciones de almdidón y enzima.}
	\begin{tabular}{cccccc}
		\toprule
		\multirow{2}[4]{*}{Tiempo (min)} & \multicolumn{2}{p{10.415em}}{\centering Pululanasa (1U/mL) Almidón 1.5\% (m/v)} &       & \multicolumn{2}{p{10em}}{\centering Pululanasa(0.5 U/mL)  \\ Almidón 1\% (m/v)} \\
		\cmidrule{2-3}\cmidrule{5-6}          & Prom. Abs.  & C.V. \% &       & Prom. Abs.  & C.V. \% \\
		\midrule
		0     & 0.281 & 0.36  &       & 0.281 & 2.67 \\
		1     & 0.399 & 2.25  &       & 0.324 & 6.44 \\
		2     & 0.408 & 3.68  &       & 0.358 & 0.58 \\
		3     & 0.387 & 5.08  &       & 0.392 & 3.46 \\
		4     & 0.401 & 4.78  &       & 0.410 & 2.13 \\
		5     & 0.418 & 2.58  &       & 0.389 & 4.25 \\
		6     & 0.446 & 0.72  &       & 0.441 & 6.90 \\
		7     & 0.470 & 2.71  &       & 0.417 & 2.72 \\
		8     & 0.470 & 2.71  &       & 0.351 & 4.94 \\
		9     & 0.453 & 2.60  &       & 0.305 & 2.56 \\
		10    & 0.402 & 3.46  &       & 0.311 & 1.22 \\
		\bottomrule
	\end{tabular}%
	\label{tab:b1}%
\end{table}%

% Table generated by Excel2LaTeX from sheet 'Hoja1'
\begin{table}[H]
	\centering
	\caption{Curvas de progreso de pululanasa a diferentes temperaturas.}
	\begin{tabular}{ccccccccc}
		\toprule
		\multirow{2}[4]{*}{Tiempo (min)} & \multicolumn{2}{c}{Temperatura 45ºC} &       & \multicolumn{2}{c}{Temperatura 50ºC} &       & \multicolumn{2}{c}{Temperatura 55ºC} \\
		\cmidrule{2-3}\cmidrule{5-6}\cmidrule{8-9}          & Prom. Abs. & C.V. \% &       & Prom. Abs. & C.V. \% &       & Prom. Abs. & C.V. \% \\
		\midrule
		0     & 0.281 & 3.04  &       & 0.281 & 1.23  &       & 0.281 & 0.62 \\
		2     & 0.353 & 3.43  &       & 0.426 & 5.99  &       & 0.330 & 5.33 \\
		4     & 0.372 & 3.62  &       & 0.439 & 4.58  &       & 0.365 & 2.13 \\
		6     & 0.383 & 2.37  &       & 0.429 & 5.12  &       & 0.404 & 1.31 \\
		8     & 0.354 & 2.96  &       & 0.420 & 3.47  &       & 0.423 & 1.65 \\
		10    & 0.367 & 1.50  &       & 0.362 & 4.76  &       & 0.399 & 3.26 \\
		12    & 0.363 & 2.38  &       & 0.375 & 7.66  &       & 0.450 & 3.93 \\
		14    & 0.352 & 2.81  &       & 0.367 & 2.18  &       & 0.420 & 1.75 \\
		16    & 0.373 & 1.39  &       & 0.422 & 4.39  &       & 0.348 & 1.52 \\
		18    & 0.368 & 4.15  &       & 0.433 & 4.32  &       & 0.308 & 2.16 \\
		20    & 0.379 & 1.45  &       & 0.434 & 6.81  &       & 0.315 & 2.22 \\
		\bottomrule
	\end{tabular}%
	\label{tab:b2}%
\end{table}%

% Table generated by Excel2LaTeX from sheet 'Hoja1'
\begin{table}[H]
	\centering
	\caption{Curvas de progreso de pululanasa a diferentes pH.}
	\begin{tabular}{ccccccccc}
		\toprule
		\multirow{2}[4]{*}{Tiempo (min)} & \multicolumn{2}{c}{ pH 5} &       & \multicolumn{2}{c}{pH 6} &       & \multicolumn{2}{c}{pH 7} \\
		\cmidrule{2-3}\cmidrule{5-6}\cmidrule{8-9}          & Prom. Abs. & C.V. \% &       & Prom. Abs. & C.V. \% &       & Prom. Abs. & C.V. \% \\
		\midrule
		0     & 0.281 & 9.57  &       & 0.281 & 1.23  &       & 0.164 & 2.66 \\
		2     & 0.351 & 2.70  &       & 0.426 & 5.99  &       & 0.189 & 3.00 \\
		4     & 0.363 & 3.51  &       & 0.439 & 4.58  &       & 0.208 & 3.82 \\
		6     & 0.406 & 2.09  &       & 0.429 & 5.12  &       & 0.213 & 0.94 \\
		8     & 0.400 & 5.38  &       & 0.420 & 3.47  &       & 0.218 & 1.48 \\
		10    & 0.430 & 6.85  &       & 0.362 & 4.76  &       & 0.228 & 2.64 \\
		12    & 0.469 & 3.63  &       & 0.375 & 7.66  &       & 0.221 & 0.78 \\
		14    & 0.457 & 3.28  &       & 0.367 & 2.18  &       & 0.238 & 1.94 \\
		16    & 0.503 & 2.93  &       & 0.422 & 4.39  &       & 0.226 & 1.42 \\
		18    & 0.537 & 5.77  &       & 0.433 & 4.32  &       & 0.236 & 2.14 \\
		20    & 0.532 & 1.79  &       & 0.434 & 6.81  &       & 0.237 & 6.33 \\
		\bottomrule
	\end{tabular}%
	\label{tab:b3}%
\end{table}%

% Table generated by Excel2LaTeX from sheet 'Hoja1'
\begin{table}[H]
	\centering
	\caption{Curva de progreso de pululanasa a pH 3.5 utilizando yodo y DNS para revelar las muestras.}
		\begin{adjustbox}{width=\textwidth}
	\begin{tabular}{cccccccccccc}
		\toprule
		\multirow{3}[6]{*}{Tiempo (min)} & \multicolumn{5}{c}{Control}           &       & \multicolumn{5}{c}{Pululanasa (0.5U/mL)} \\
		\cmidrule{2-6}\cmidrule{8-12}          & \multicolumn{2}{c}{DNS} &       & \multicolumn{2}{c}{Yodo} &       & \multicolumn{2}{c}{DNS} &       & \multicolumn{2}{c}{Yodo} \\
		\cmidrule{2-3}\cmidrule{5-6}\cmidrule{8-9}\cmidrule{11-12}          & Prom. Abs. & C.V. \% &       & Prom. Abs. & C.V. \% &       & Prom. Abs. & C.V. \% &       & Prom. Abs. & C.V. \% \\
		\midrule
		0     & 0.093 & 1.08  &       & 0.017 & 5.88  &       & 0.121 & 2.19  &       & 0.014 & 4.22 \\
		10    & 0.096 & 3.61  &       & 0.017 & 3.46  &       & 0.254 & 5.30  &       & 0.012 & 0.00 \\
		20    & 0.097 & 1.19  &       & 0.014 & 0.00  &       & 0.411 & 5.84  &       & 0.013 & 0.00 \\
		30    & 0.100 & 1.53  &       & 0.015 & 3.77  &       & 0.705 & 0.16  &       & 0.012 & 0.00 \\
		40    & 0.097 & 1.57  &       & 0.015 & 3.94  &       & 0.964 & 1.04  &       & 0.019 & 2.99 \\
		50    & 0.098 & 1.77  &       & 0.014 & 4.22  &       & 1.204 & 4.06  &       & 0.020 & 0.00 \\
		60    & 0.100 & 1.41  &       & 0.011 & 5.41  &       & 1.531 & 3.69  &       & 0.020 & 0.00 \\
		\bottomrule
	\end{tabular}%
	\end{adjustbox}
	\label{tab:b4}%
\end{table}%

% Table generated by Excel2LaTeX from sheet 'exp'
\begin{table}[H]
	\centering
	\caption{Curva de progreso de pululanasa a pH 5.0 utilizando yodo y DNS para revelar las muestras.}
	\begin{adjustbox}{width=\textwidth}
	\begin{tabular}{ccccccccccc}
		\toprule
		\multirow{3}[6]{*}{Tiempo (min)} & \multicolumn{5}{c}{Control}           &       & \multicolumn{4}{c}{Pululanasa (0.5U/mL)} \\
		\cmidrule{2-6}\cmidrule{8-11}          & \multicolumn{2}{c}{DNS} &       & \multicolumn{2}{c}{Yodo} &       & \multicolumn{2}{c}{DNS} & \multicolumn{2}{c}{Yodo} \\
		\cmidrule{2-3}\cmidrule{5-6}\cmidrule{8-11}          & Prom. Abs. & C.V. \% &       & Prom. Abs. & C.V. \% &       & Prom. Abs. & C.V. \% & Prom. Abs. & C.V. \% \\
		\midrule
		0     & 0.122 & 0.82  &       & 0.351 & 0.28  &       & 0.139 & 2.20  & 0.371 & 2.88 \\
		10    & 0.124 & 0.93  &       & 0.448 & 1.02  &       & 0.170 & 0.90  & 0.471 & 10.72 \\
		20    & 0.122 & 0.47  &       & 0.451 & 6.21  &       & 0.196 & 3.12  & 0.678 & 1.21 \\
		30    & 0.121 & 2.65  &       & 0.370 & 0.87  &       & 0.263 & 6.03  & 0.810 & 1.69 \\
		40    & 0.126 & 4.80  &       & 0.415 & 0.48  &       & 0.320 & 5.03  & 0.941 & 2.99 \\
		50    & 0.122 & 0.94  &       & 0.427 & 2.85  &       & 0.408 & 4.92  & 1.044 & 2.14 \\
		60    & -     & -     &       & 0.476 & 4.68  &       & 0.514 & 6.84  & 1.211 & 2.79 \\
		\bottomrule
	\end{tabular}%
	\end{adjustbox}
	\label{tab:b5}%
\end{table}%

\subsection{Experimentos con isoamilasa}

% Table generated by Excel2LaTeX from sheet 'Hoja1'
\begin{table}[H]
	\centering
	\caption{Curvas de progreso de isoamilasa a diferentes concentraciones de enzima}
	\begin{adjustbox}{width=\textwidth}
	\begin{tabular}{cccccccp{5em}cccc}
		\toprule
		\multicolumn{1}{c}{\multirow{2}[4]{*}{Tiempo (min}} & \multicolumn{2}{c}{Control} &       & \multicolumn{2}{c}{Isoamilasa 250 U/mL} &       & \multicolumn{2}{c}{Control} &       & \multicolumn{2}{c}{Isoamilasa 2,500 U/mL} \\
		\cmidrule{2-3}\cmidrule{5-6}\cmidrule{8-9}\cmidrule{11-12}          & Prom. Abs. & CV\%  &       & Prom. Abs. & CV\%  &       & \multicolumn{1}{c}{Prom. Abs.} & CV\%  &       & Prom. Abs. & CV\% \\
		\midrule
		0     & 0.15  & 9.35  &       & 0     & 26.65 &       & \multicolumn{1}{c}{0.13} & 12.59 &       & 0.14  & 7.91 \\
		5     & 0.14  & 0.43  &       & 0.01  & 19.25 &       & \multicolumn{1}{c}{0.13} & 1.63  &       & 0.14  & 6.48 \\
		10    & 0.14  & 0.82  &       & 0.04  & 12.62 &       & \multicolumn{1}{c}{0.11} & 2.85  &       & 0.15  & 2.09 \\
		15    & 0.22  & 1.49  &       & 0.21  & 5.96  &       & \multicolumn{1}{c}{0.1} & 2.99  &       & 0.21  & 13.45 \\
		20    & 0.21  & 0.98  &       & 0.07  & 8.66  &       & \multicolumn{1}{c}{0.1} & 1.55  &       & 0.22  & 3.59 \\
		25    & 0.18  & 0.65  &       & 0.06  & 1.69  &       & \multicolumn{1}{c}{0.1} & 2.27  &       & 0.28  & 3.75 \\
		30    & 0.15  & 4.20  &       & 0.04  & 3.24  &       & \centering-                                  & \centering-     &       & 0.32  & 2.27 \\
		\bottomrule
	\end{tabular}%
	\end{adjustbox}
	\label{tab:c1}%
\end{table}%

% Table generated by Excel2LaTeX from sheet 'Hoja1'
\begin{table}[H]
	\centering
	\caption{Curvas de progreso de isoamilasa a diferentes temperaturas.}
	\begin{adjustbox}{width=\textwidth}
	\begin{tabular}{cccccccccccc}
		\toprule
		\multicolumn{1}{c}{\multirow{3}[6]{*}{Tiempo (min)}} & \multicolumn{5}{c}{Temperatura 30ºC}  &       & \multicolumn{5}{c}{Temperatura 35ºC} \\
		\cmidrule{2-6}\cmidrule{8-12}          & \multicolumn{2}{c}{Control} &       & \multicolumn{2}{c}{Isoamilasa (2,500 U/mL)} &       & \multicolumn{2}{c}{Control} &       & \multicolumn{2}{c}{Isoamilasa (2,500 U/mL)} \\
		\cmidrule{2-3}\cmidrule{5-6}\cmidrule{8-9}\cmidrule{11-12}          & Prom. Abs. & CV\%  &       & Prom. Abs. & CV\%  &       & Prom. Abs. & CV\%  &       & Prom. Abs. & CV\% \\
		\midrule
		0     & 0.163 & 2.47  &       & 0.166 & 11.19 &       & 0.130 & 2.37  &       & 0.120 & 0.47 \\
		10    & 0.159 & 1.31  &       & 0.149 & 1.03  &       & 0.130 & 0.45  &       & 0.130 & 0.78 \\
		20    & 0.16  & 0.72  &       & 0.147 & 0.39  &       & 0.130 & 2.36  &       & 0.130 & 1.79 \\
		30    & 0.161 & 1.08  &       & 0.15  & 1.68  &       & 0.130 & 6.18  &       & 0.130 & 1.77 \\
		40    & 0.158 & 0.37  &       & 0.154 & 0.65  &       & 0.130 & 1.93  &       & 0.140 & 1.79 \\
		50    & 0.156 & 0.64  &       & 0.162 & 0.36  &       & 0.130 & 3.51  &       & 0.150 & 9.14 \\
		60    & 0.159 & 5.03  &       & 0.166 & 0.35  &       & 0.130 & 1.82  &       & 0.160 & 6.39 \\
		\bottomrule
	\end{tabular}%
	\end{adjustbox}
	\label{tab:c2}%
\end{table}%


% Table generated by Excel2LaTeX from sheet 'exp'
\begin{table}[H]
	\centering
	\caption{Curvas de progreso de isoamilasa a diferentes temperaturas.}
	\begin{adjustbox}{width=\textwidth}
	\begin{tabular}{ccccccccccc}
		\toprule
		\multicolumn{1}{c}{\multirow{3}[6]{*}{Tiempo (min)}} & \multicolumn{5}{c}{Temperatura 40ºC}  &       & \multicolumn{4}{c}{Temperatura 60ºC} \\
		\cmidrule{2-6}\cmidrule{8-11}          & \multicolumn{2}{c}{Control} &       & \multicolumn{2}{c}{Isoamilasa (2,500 U/mL)} &       & \multicolumn{2}{c}{Control} & \multicolumn{2}{c}{Isoamilasa (2,500 U/mL)} \\
		\cmidrule{2-3}\cmidrule{5-6}\cmidrule{8-11}          & Prom. Abs. & CV\%  &       & Prom. Abs. & CV\%  &       & Prom Abs. & CV\%  & Prom Abs. & CV\% \\
		\midrule
		0     & 0.120 & 3.74  &       & 0.160 & 1.08  &       & 0.102 & 0.00  & 0.098 & 0.59 \\
		10    & 0.130 & 3.01  &       & 0.220 & 0.79  &       & 0.111 & 0.90  & 0.096 & 1.21 \\
		20    & 0.130 & 1.29  &       & 0.280 & 0.91  &       & 0.119 & 0.00  & 0.098 & 1.17 \\
		30    & 0.150 & 1.01  &       & 0.310 & 0.49  &       & 0.132 & 0.76  & 0.111 & 0.90 \\
		40    & 0.120 & 0.96  &       & 0.360 & 2.41  &       & 0.142 & 1.41  & 0.109 & 2.66 \\
		50    & 0.110 & 1.82  &       & 0.380 & 2.05  &       & 0.158 & 7.27  & 0.122 & 4.50 \\
		60    & 0.100 & 1.55  &       & 0.410 & 0.51  &       & 0.196 & 1.64  & 0.123 & 2.34 \\
		\bottomrule
	\end{tabular}%
	\end{adjustbox}
	\label{tab:c3}%
\end{table}%


% Table generated by Excel2LaTeX from sheet 'Hoja1'
\begin{table}[H]
	\centering
	\caption{Curvas de progreso de isoamilasa utilizando diferentes concentraciones de glucosa como inhibidor.}
		\begin{adjustbox}{width=\textwidth}
	\begin{tabular}{cccccccccccc}
		\toprule
		\multicolumn{1}{c}{\multirow{3}[6]{*}{Tiempo (min)}} & \multicolumn{5}{c}{Glucosa 0.1 mg/mL} &       & \multicolumn{5}{c}{Glucosa 0.2 mg/mL} \\
		\cmidrule{2-6}\cmidrule{8-12}          & \multicolumn{2}{c}{DNS} &       & \multicolumn{2}{c}{Yodo} &       & \multicolumn{2}{c}{DNS} &       & \multicolumn{2}{c}{Yodo} \\
		\cmidrule{2-3}\cmidrule{5-6}\cmidrule{8-9}\cmidrule{11-12}          & Prom Abs. & CV\%  &       & Prom Abs. & CV\%  &       & Prom Abs. & CV\%  &       & Prom Abs. & CV\% \\
		\midrule
		0     & 0.191 & 1.09  &       & 0.311 & 5.17  &       & 0.322 & 1.90  &       & 0.237 & 3.27 \\
		10    & 0.223 & 2.63  &       & 0.503 & 9.57  &       & 0.386 & 1.58  &       & 0.469 & 4.16 \\
		20    & 0.247 & 3.66  &       & 0.599 & 6.18  &       & 0.391 & 3.48  &       & 0.635 & 5.35 \\
		30    & 0.278 & 3.79  &       & 0.770 & 1.01  &       & 0.439 & 5.97  &       & 0.660 & 4.10 \\
		40    & 0.315 & 13.17 &       & 0.880 & 1.58  &       & 0.452 & 2.77  &       & 0.933 & 5.24 \\
		50    & 0.332 & 7.33  &       & 1.045 & 1.13  &       & 0.491 & 1.69  &       & 1.077 & 1.16 \\
		\bottomrule
	\end{tabular}%
	\end{adjustbox}
	\label{tab:c4}%
\end{table}%



\subsection{Modificación enzimática de sémola de \textit{Triticum durum}}

\subsubsection{Experimento 1}
% Table generated by Excel2LaTeX from sheet 'Hoja1'
\begin{table}[H]
	\centering
	\caption{Curva de progreso de pululanasa en sémola de \textit{Triticum durum} para experimento 1.}
		\begin{adjustbox}{width=\textwidth}
	\begin{tabular}{cccccccccccc}
		\toprule
		\multicolumn{1}{c}{\multirow{3}[6]{*}{Tiempo (min)}} & \multicolumn{5}{c}{Control}           &       & \multicolumn{5}{c}{Pululanasa (0.5U/mL)} \\
		\cmidrule{2-6}\cmidrule{8-12}          & \multicolumn{2}{c}{DNS} &       & \multicolumn{2}{c}{Yodo} &       & \multicolumn{2}{c}{DNS} &       & \multicolumn{2}{c}{Yodo} \\
		\cmidrule{2-3}\cmidrule{5-6}\cmidrule{8-9}  \cmidrule{11-12}         & Prom. Abs. & C.V. \% &       & Prom. Abs. & C.V. \% &       & Prom. Abs. & C.V. \% &       & Prom. Abs. & C.V. \% \\
		\midrule
		0     & 0.099 & 1.01  &       & 0.130 & 11.52 &       & 0.155 & 32.10 &       & 0.385 & 4.34 \\
		20    & 0.167 & 12.21 &       & 0.142 & 4.93  &       & 0.122 & 6.84  &       & 0.535 & 2.73 \\
		40    & 0.100 & 2.90  &       & 0.169 & 2.40  &       & 0.137 & 2.34  &       & 0.746 & 3.47 \\
		\bottomrule
	\end{tabular}%
	\end{adjustbox}
	\label{tab:e1.1}%
\end{table}%

% Table generated by Excel2LaTeX from sheet 'Hoja1'
\begin{table}[H]
	\centering
	\caption{Curva de progreso para experimento 1 de isoamilasa en sémola de \textit{Triticum durum} pretratada con pululanasa.}
		\begin{adjustbox}{width=\textwidth}
	\begin{tabular}{cccccccccccc}
		\toprule
		\multicolumn{1}{c}{\multirow{3}[6]{*}{Tiempo (min)}} & \multicolumn{5}{c}{Control}           &       & \multicolumn{5}{c}{Isoamilasa (2,500 U/mL)} \\
		\cmidrule{2-6}\cmidrule{8-12}          & \multicolumn{2}{c}{DNS} &       & \multicolumn{2}{c}{Yodo} &       & \multicolumn{2}{c}{DNS} &       & \multicolumn{2}{c}{Yodo} \\
		\cmidrule{2-3}\cmidrule{5-6}\cmidrule{8-9}\cmidrule{11-12}          & Prom. Abs. & C.V. \% &       & Prom. Abs. & C.V. \% &       & Prom. Abs. & C.V. \% &       & Prom. Abs. & C.V. \% \\
		\midrule
		0     & 0.103 & 2.03  &       & 0.108 & 0.93  &       & 0.151 & 0.66  &       & 0.747 & 7.53 \\
		10    & 0.100 & 2.51  &       & 0.094 & 1.22  &       & 0.143 & 6.92  &       & 0.839 & 4.72 \\
		20    & 0.101 & 1.71  &       & 0.104 & 0.55  &       & 0.146 & 2.41  &       & 0.826 & 6.11 \\
		30    & 0.101 & 2.62  &       & 0.102 & 3.14  &       & 0.148 & 2.44  &       & 1.027 & 8.60 \\
		40    & 0.098 & 2.04  &       & 0.105 & 2.40  &       & 0.150 & 0.67  &       & 0.880 & 5.24 \\
		50    & 0.095 & 4.24  &       & 0.103 & 1.48  &       & 0.163 & 2.84  &       & 0.929 & 6.49 \\
		60    & 0.099 & 0.59  &       & 0.099 & 3.84  &       & 0.163 & 3.39  &       & 0.885 & 3.67 \\
		\bottomrule
	\end{tabular}%
	\end{adjustbox}
	\label{tab:e1.2}%
\end{table}%

% Table generated by Excel2LaTeX from sheet 'Hoja1'

% Table generated by Excel2LaTeX from sheet 'Hoja1'
\begin{table}[htbp]
	\centering
	\caption{Índice glicémico del experimento 1 para la de sémola de textit{Triticum durum} modificada con pululanasa e isoamilasa.}
		\begin{adjustbox}{width=\textwidth}
	\begin{tabular}{cccccccccccc}
		\toprule
		\multicolumn{1}{c}{\multirow{3}[6]{*}{Tiempo (min)}} & \multicolumn{5}{c}{Control}           &       & \multicolumn{5}{c}{Sémola modificada} \\
		\cmidrule{2-6}\cmidrule{8-12}          & \multicolumn{2}{c}{DNS} &       & \multicolumn{2}{c}{Yodo} &       & \multicolumn{2}{c}{DNS} &       & \multicolumn{2}{c}{Yodo} \\
		\cmidrule{2-3}\cmidrule{5-6}\cmidrule{8-9}\cmidrule{11-12}          & Prom. Abs. & C.V. \% &       & Prom. Abs. & C.V. \% &       & Prom. Abs. & C.V. \% &       & Prom. Abs. & C.V. \% \\
		\midrule
		0     & 0.094 & 1.84  &       & 0.107 & 3.90  &       & 0.110 & 1.38  &       & 0.340 & 19.79 \\
		20    & 0.944 & 10.27 &       & 1.239 & 9.41  &       & 0.587 & 8.20  &       & 0.358 & 4.50 \\
		40    & 1.498 & 22.24 &       & 1.235 & 7.50  &       & 0.544 & 1.80  &       & 0.364 & 3.70 \\
		60    & 1.824 & 0.00  &       & 1.395 & 5.63  &       & 0.727 & 2.19  &       & 0.372 & 3.92 \\
		\bottomrule
	\end{tabular}%
	\end{adjustbox}
	\label{tab:e1.3}%
\end{table}%



\subsubsection{Experimento 2}

% Table generated by Excel2LaTeX from sheet 'Hoja1'
\begin{table}[H]
	\centering
	\caption{Curva de progreso de pululanasa en sémola de \textit{Triticum durum} para experimento 2.}
		\begin{adjustbox}{width=\textwidth}
	\begin{tabular}{cccccccccccc}
		\toprule
		\multicolumn{1}{c}{\multirow{3}[6]{*}{Tiempo (min)}} & \multicolumn{5}{c}{Control}           &       & \multicolumn{5}{c}{Pululanasa (0.5U/mL)} \\
		\cmidrule{2-6}\cmidrule{8-12}          & \multicolumn{2}{c}{DNS} &       & \multicolumn{2}{c}{Yodo} &       & \multicolumn{2}{c}{DNS} &       & \multicolumn{2}{c}{Yodo} \\
		\cmidrule{2-3}\cmidrule{8-9}\cmidrule{11-12}          & Prom. Abs. & C.V. \% &       & Prom. Abs. & C.V. \% &       & Prom. Abs. & C.V. \% &       & Prom. Abs. & C.V. \% \\
		\midrule
		0     & 0.131 & 1.53  &       & 0.124 & 1.68  &       & 0.110 & 2.29  &       & 0.208 & 3.85 \\
		20    & 0.133 & 1.15  &       & 0.165 & 7.54  &       & 0.138 & 2.32  &       & 0.430 & 3.46 \\
		40    & 0.126 & 2.29  &       & 0.287 & 5.48  &       & 0.175 & 0.66  &       & 0.814 & 4.46 \\
		\bottomrule
	\end{tabular}%
	\end{adjustbox}
	\label{tab:e2.1}%
\end{table}%

% Table generated by Excel2LaTeX from sheet 'Hoja1'
\begin{table}[H]
	\centering
	\caption{Curva de progreso para experimento 2 de isoamilasa en sémola de \textit{Triticum durum} pretratada con pululanasa.}
		\begin{adjustbox}{width=\textwidth}
	\begin{tabular}{cccccccccccc}
		\toprule
		\multicolumn{1}{c}{\multirow{3}[6]{*}{Tiempo (min)}} & \multicolumn{5}{c}{Control}           &       & \multicolumn{5}{c}{Isoamilasa (2,500 U/mL)} \\
		\cmidrule{2-6}\cmidrule{8-12}          & \multicolumn{2}{c}{DNS} &       & \multicolumn{2}{c}{Yodo} &       & \multicolumn{2}{c}{DNS} &       & \multicolumn{2}{c}{Yodo} \\
		\cmidrule{2-3}\cmidrule{5-6}\cmidrule{8-9}\cmidrule{11-12}          & Prom. Abs. & C.V. \% &       & Prom. Abs. & C.V. \% &       & Prom. Abs. & C.V. \% &       & Prom. Abs. & C.V. \% \\
		\midrule
		0     & 0.126 & 1.66  &       & 0.115 & 2.79  &       & 0.137 & 2.35  &       & 0.582 & 4.43 \\
		10    & 0.124 & 0.01  &       & 0.124 & 1.24  &       & 0.146 & 1.81  &       & 0.677 & 3.53 \\
		20    & 0.122 & 0.01  &       & 0.123 & 2.04  &       & 0.152 & 4.39  &       & 0.788 & 11.93 \\
		30    & 0.126 & 0.00  &       & 0.129 & 5.83  &       & 0.160 & 2.00  &       & 0.830 & 2.99 \\
		40    & 0.128 & 0.04  &       & 0.138 & 5.23  &       & 0.172 & 3.50  &       & 0.880 & 2.51 \\
		50    & 0.128 & 0.03  &       & 0.153 & 1.88  &       & 0.178 & 4.53  &       & 0.887 & 4.67 \\
		60    & 0.128 & 0.01  &       & 0.134 & 3.10  &       & 0.178 & 2.33  &       & 0.949 & 9.55 \\
		\bottomrule
	\end{tabular}%
	\end{adjustbox}
	\label{tab:e2.2}%
\end{table}%


% Table generated by Excel2LaTeX from sheet 'Hoja1'
\begin{table}[H]
	\centering
	\caption{Índice glicémico del experimento 2 para la de sémola de textit{Triticum durum} modificada con pululanasa e isoamilasa.}
		\begin{adjustbox}{width=\textwidth}
	\begin{tabular}{cccccccccccc}
		\toprule
		\multicolumn{1}{c}{\multirow{3}[6]{*}{Tiempo (min)}} & \multicolumn{5}{c}{Control}           &       & \multicolumn{5}{c}{Sémola modificada} \\
		\cmidrule{2-6}\cmidrule{8-12}          & \multicolumn{2}{c}{DNS} &       & \multicolumn{2}{c}{Yodo} &       & \multicolumn{2}{c}{DNS} &       & \multicolumn{2}{c}{Yodo} \\
		\cmidrule{2-3}\cmidrule{5-6}\cmidrule{8-9}\cmidrule{11-12}          & Prom. Abs. & C.V. \% &       & Prom. Abs. & C.V. \% &       & Prom. Abs. & C.V. \% &       & Prom. Abs. & C.V. \% \\
		\midrule
		0     & 0.132 & 1.52  &       & 0.261 & 0.38  &       & 0.039 & 87.42 &       & 0.371 & 26.28 \\
		20    & 0.790 & 2.58  &       & 0.904 & 0.40  &       & 0.312 & 14.99 &       & 0.471 & 12.12 \\
		40    & 0.983 & 3.71  &       & 0.874 & 3.18  &       & 0.489 & 7.90  &       & 0.678 & 3.71 \\
		60    & 1.122 & 0.06  &       & 0.900 & 0.08  &       & 0.552 & 13.32 &       & 0.810 & 3.10 \\
		\bottomrule
	\end{tabular}%
	\end{adjustbox}
	\label{tab:e2.3}%
\end{table}%





\subsubsection{Experimento 3}
% Table generated by Excel2LaTeX from sheet 'Hoja1'
\begin{table}[H]
	\centering
	\caption{Curva de progreso de pululanasa en sémola de \textit{Triticum durum} para experimento 3.}
		\begin{adjustbox}{width=\textwidth}
	\begin{tabular}{cccccccccccc}
		\toprule
		\multicolumn{1}{c}{\multirow{3}[6]{*}{Tiempo (min)}} & \multicolumn{5}{c}{Control}           &       & \multicolumn{5}{c}{Pululanasa (0.5U/mL)} \\
		\cmidrule{2-6}\cmidrule{8-12}          & \multicolumn{2}{c}{DNS} &       & \multicolumn{2}{c}{Yodo} &       & \multicolumn{2}{c}{DNS} &       & \multicolumn{2}{c}{Yodo} \\
		\cmidrule{2-3}\cmidrule{5-6}\cmidrule{8-9}\cmidrule{11-12}          & Prom. Abs. & C.V. \% &       & Prom. Abs. & C.V. \% &       & Prom. Abs. & C.V. \% &       & Prom. Abs. & C.V. \% \\
		\midrule
		0     & 0.116 & 3.04  &       & 0.213 & 6.70  &       & 0.098 & 0.59  &       & 0.246 & 0.24 \\
		20    & 0.114 & 1.34  &       & 0.189 & 3.96  &       & 0.124 & 2.91  &       & 0.379 & 1.60 \\
		40    & 0.115 & 2.00  &       & 0.265 & 13.27 &       & 0.146 & 1.43  &       & 0.717 & 0.79 \\
		\bottomrule
	\end{tabular}%
	\end{adjustbox}
	\label{tab:e3.1}%
\end{table}%

% Table generated by Excel2LaTeX from sheet 'Hoja1'
\begin{table}[H]
	\centering
	\caption{Curva de progreso para experimento 3 de isoamilasa en sémola de \textit{Triticum durum} pretratada con pululanasa.}
		\begin{adjustbox}{width=\textwidth}
	\begin{tabular}{cccccccccccc}
		\toprule
		\multicolumn{1}{c}{\multirow{3}[6]{*}{Tiempo (min)}} & \multicolumn{5}{c}{Control}           &       & \multicolumn{5}{c}{Isoamilasa (2,500 U/mL)} \\
		\cmidrule{2-6}\cmidrule{8-12}          & \multicolumn{2}{c}{DNS} &       & \multicolumn{2}{c}{Yodo} &       & \multicolumn{2}{c}{DNS} &       & \multicolumn{2}{c}{Yodo} \\
		\cmidrule{2-3}\cmidrule{5-6}\cmidrule{8-9}\cmidrule{11-12}          & Prom. Abs. & C.V. \% &       & Prom. Abs. & C.V. \% &       & Prom. Abs. & C.V. \% &       & Prom. Abs. & C.V. \% \\
		\midrule
		0     & 0.198 & 1.27  &       & 0.160 & 20.40 &       & 0.126 & 0.79  &       & 0.698 & 0.58 \\
		10    & 0.192 & 4.07  &       & 0.163 & 7.44  &       & 0.144 & 2.81  &       & 0.725 & 2.08 \\
		20    & 0.191 & 2.36  &       & 0.264 & 22.17 &       & 0.174 & 3.92  &       & 0.839 & 3.82 \\
		30    & 0.172 & 14.26 &       & 0.349 & 5.61  &       & 0.161 & 3.80  &       & 0.874 & 4.60 \\
		40    & 0.108 & 4.58  &       & 0.304 & 3.19  &       & 0.169 & 2.73  &       & 0.861 & 2.67 \\
		50    & 0.110 & 1.38  &       & 0.286 & 6.46  &       & 0.200 & 2.08  &       & 0.924 & 0.69 \\
		60    & 0.112 & 0.51  &       & 0.275 & 4.24  &       & 0.207 & 0.84  &       & 0.961 & 3.88 \\
		\bottomrule
	\end{tabular}%
	\end{adjustbox}
	\label{tab:e3.2}%
\end{table}%

% Table generated by Excel2LaTeX from sheet 'Hoja1'
\begin{table}[H]
	\centering
	\caption{Índice glicémico del experimento 3 para la de sémola de \textit{Triticum durum} modificada con pululanasa e isoamilasa.}
		\begin{adjustbox}{width=\textwidth}
	\begin{tabular}{cccccccccccc}
		\toprule
		\multicolumn{1}{c}{\multirow{3}[6]{*}{Tiempo (min)}} & \multicolumn{5}{c}{Control}           &       & \multicolumn{5}{c}{Sémola modificada} \\
		\cmidrule{2-6}\cmidrule{8-12}          & \multicolumn{2}{c}{DNS} &       & \multicolumn{2}{c}{Yodo} &       & \multicolumn{2}{c}{DNS} &       & \multicolumn{2}{c}{Yodo} \\
		\cmidrule{2-3}\cmidrule{5-6}\cmidrule{8-9}\cmidrule{11-12}          & Prom. Abs. & C.V. \% &       & Prom. Abs. & C.V. \% &       & Prom. Abs. & C.V. \% &       & Prom. Abs. & C.V. \% \\
		\midrule
		0     & 0.111 & 0.90  &       & 0.104 & 1.67  &       & 0.123 & 1.88  &       & 0.174 & 23.08 \\
		20    & 0.720 & 8.49  &       & 0.756 & 7.98  &       & 0.598 & 1.60  &       & 0.282 & 6.05 \\
		40    & 1.472 & 5.93  &       & 0.829 & 5.81  &       & 0.709 & 6.39  &       & 0.260 & 11.87 \\
		60    & 1.521 & 10.13 &       & 0.835 & 0.00  &       & 0.743 & 19.60 &       & 0.240 & 0.00 \\
		\bottomrule
	\end{tabular}%
	\end{adjustbox}
	\label{tab:e3.3}%
\end{table}%



\section{Análisis de resultados}


\begin{figure}[H]
	\centering
	\begin{tikzpicture}
	\begin{axis}
	[%xtick={0,5,10,15,20},
	xmin=-0.4,
  	xmax=20.4,
	width=\textwidth,
	height=\axisdefaultheight,
	 xlabel={Tiempo (min)},ylabel={Prom. Abs. (550 nm)},legend pos=north west, legend columns=1]
	
	\addplot[black,mark=triangle*] table[x=a, y=b, col sep=comma] {1.1.csv};\addlegendentry{$\alpha$-amilasa (3.985 U/mL)}
	\addplot[black,mark=o] table [x=a, y=c, col sep=comma] {1.1.csv};\addlegendentry{$\alpha$-amilasa (7.97 U/mL)}
	\end{axis}
	\end{tikzpicture}
	
	\caption{Gráfica de los datos observados en la Tabla \ref{tab:a1}. }
	\label{g1.1}
\end{figure}

\begin{figure}[H]
	\centering

	\begin{tikzpicture}
	\begin{axis}
	[%xtick={0,5,10,15,20},
		xmin=-0.4,
	xmax=20.4,
	width=\textwidth,
	height=\axisdefaultheight,
	xlabel={Tiempo (min)},ylabel={Prom. Abs. (550 nm)},legend style={at={(0.5,-0.2)},anchor=north},legend style={text width=1.6cm, minimum height=1cm},, legend columns=7]
	
	\addplot[black,mark=triangle*] table[x=a, y=b, col sep=comma] {1.2.csv};\addlegendentry{Temp. 30ºC }
	\addplot[black,mark=o] table [x=a, y=c, col sep=comma] {1.2.csv};\addlegendentry{Temp. 35ºC}
	\addplot[black,mark=x] table[x=a, y=c, col sep=comma] {1.2.csv};\addlegendentry{Temp. 40ºC}
	\addplot[black,mark=triangle ] table[x=a, y=d, col sep=comma] {1.2.csv};\addlegendentry{Temp. 45ºC}
	\addplot[black,mark=diamond*] table[x=a, y=e, col sep=comma] {1.2.csv};\addlegendentry{Temp. 50ºC}
	\addplot[black,mark=star] table[x=a, y=f, col sep=comma] {1.2.csv};\addlegendentry{Temp. 55ºC}
	\addplot[black,mark=square*] table[x=a, y=g, col sep=comma] {1.2.csv};\addlegendentry{Temp. 60ºC}
	\end{axis}
	\end{tikzpicture}
	\caption{Gráfica de los datos observados en las Tablas \ref{tab:a2} y \ref{tab:a3}. }
	\label{g1.2}
\end{figure}

\begin{figure}[H]
	\centering
	\begin{tikzpicture}
	\begin{axis}
	[%xtick={0,5,10,15,20},
	xmin=-8,
	xmax=308,
	width=\textwidth,
	height=\axisdefaultheight,
	xlabel={Tiempo (min)},ylabel={Prom. Abs. (550 nm)},legend pos=north west, legend columns=1]
	
	\addplot[black,mark=triangle*] table[x=a, y=b, col sep=comma] {1.3.csv};\addlegendentry{Almidón 0.5\% (m/v)}
	\addplot[black,mark=o] table [x=a, y=c, col sep=comma] {1.3.csv};\addlegendentry{Sémola 0.5\% (m/v)}

	\end{axis}
	\end{tikzpicture}
	
	\caption{Gráfica de los datos observados en la Tabla \ref{tab:a4}. }
	\label{g1.3}
\end{figure}


\begin{figure}[H]
	\centering
	\begin{tikzpicture}
	\begin{axis}[
	xmin=-1,
	xmax=61,
	width=\textwidth,
	height=\axisdefaultheight,
	xlabel={Tiempo (min)},ylabel={Prom. Abs. (610 nm)},legend pos=north west, legend columns=1]
	\addplot[black,mark=triangle*] table[x=a, y=c, col sep=comma] {1.4.csv};\addlegendentry{Control}
	\addplot[black,mark=o] table [x=a, y=b, col sep=comma] {1.4.csv};\addlegendentry{$\alpha$-amilasa (15.94 U/mL)}
	\end{axis}
	\end{tikzpicture}
	
	\caption{Gráfica de los datos observados en la Tabla \ref{tab:a5}. }
	\label{g1.4}
\end{figure}

\begin{figure}[H]
	\centering
	\begin{tikzpicture}
	\begin{axis}
	[
	xmin=-0.3,
	xmax=10.3,
	width=\textwidth,
	height=\axisdefaultheight,
	xlabel={Tiempo (min)},ylabel={Prom. Abs. (550 nm)},legend style={at={(0.5,-0.2)},anchor=north},legend columns=2,
	legend style={/tikz/every even column/.append style={column sep=0.3cm}}]
	
	\addplot[black,mark=triangle*] table[x=a, y=b, col sep=comma] {2.1.csv};\addlegendentry{Pululanasa (1 U/mL) y Almidón 1.5\% (m/v)}
	\addplot[black,mark=o] table [x=a, y=c, col sep=comma] {2.1.csv};\addlegendentry{Pululanasa (0.5 U/mL) y Almidón 1\% (m/v)}
	\end{axis}
	\end{tikzpicture}
	
	\caption{Gráfica de los datos observados en la Tabla \ref{tab:b1}. }
	\label{g2.1}
\end{figure}

\begin{figure}[H]
	\centering
	\begin{tikzpicture}
	\begin{axis}
	[
	xmin=-0.4,
	xmax=20.4,
	width=\textwidth,
	height=\axisdefaultheight,
	xlabel={Tiempo (min)},ylabel={Prom. Abs. (550 nm)},legend style={at={(0.5,-0.2)},anchor=north},legend columns=3,
	legend style={/tikz/every even column/.append style={column sep=0.3cm}}]
	
	
	\addplot[black,mark=o] table [x=a, y=b, col sep=comma] {2.2.csv};\addlegendentry{Temp 45ºC}
	\addplot[black,mark=triangle*] table [x=a, y=c, col sep=comma] {2.2.csv};\addlegendentry{Temp 50ºC}
	\addplot[black,mark=square*] table [x=a, y=d, col sep=comma] {2.2.csv};\addlegendentry{Temp 55ºC}
	\end{axis}
	\end{tikzpicture}
	
	\caption{Gráfica de los datos observados en la Tabla \ref{tab:b2}. }
	\label{g2.2}
\end{figure}

\begin{figure}[H]
	\centering
	\begin{tikzpicture}
	\begin{axis}
	[
	xmin=-0.4,
	xmax=20.4,
	width=\textwidth,
	height=\axisdefaultheight,
	xlabel={Tiempo (min)},ylabel={Prom. Abs. (550 nm)},legend pos=north west,legend columns=3,
		legend style={/tikz/every even column/.append style={column sep=0.3cm}}]
	
	
	\addplot[black,mark=o] table [x=a, y=b, col sep=comma] {2.3.csv};\addlegendentry{pH 5}
	\addplot[black,mark=triangle*] table [x=a, y=c, col sep=comma] {2.3.csv};\addlegendentry{pH 6}
	\addplot[black,mark=square*] table [x=a, y=d, col sep=comma] {2.3.csv};\addlegendentry{pH 7}
	\end{axis}
	\end{tikzpicture}
	
	\caption{Gráfica de los datos observados en la Tabla \ref{tab:b3}. }
	\label{g2.3}
\end{figure}

\begin{figure}[H]
	\centering
	\begin{tikzpicture}
	\begin{axis}
	[
	xmin=-1,
	xmax=61,
	width=\textwidth,
	height=\axisdefaultheight,
	xlabel={Tiempo (min)},ylabel={Prom. Abs. (550 nm)},	legend style={at={(0.5,-0.2)},anchor=north},legend columns=2,
		legend style={/tikz/every even column/.append style={column sep=0.3cm}}]
	
	\addplot[black,mark=o] table [x=a, y=b, col sep=comma] {2.4.csv};\addlegendentry{Control pH 3.5}
	\addplot[black,mark=triangle*] table [x=a, y=c, col sep=comma] {2.4.csv};\addlegendentry{Pululanasa (0.5U/mL) pH 3.5}	\addplot[black,mark=square*] table [x=a, y=d, col sep=comma] {2.4.csv};\addlegendentry{Control pH 5}
	\addplot[black,mark=diamond*] table[x=a, y=e, col sep=comma] {2.4.csv};\addlegendentry{Pululanasa (0.5U/mL) pH 5}
	\end{axis}
	\end{tikzpicture}
	
	\caption{Gráfica de los datos de muestras reveladas con DNS observados en las Tabla \ref{tab:b4} y \ref{tab:b5}. }
	\label{g2.4}
\end{figure}

\begin{figure}[H]
	\centering
	\begin{tikzpicture}
	\begin{axis}
	[
	xmin=-1,
	xmax=61,
	ymax=1.6,
	width=\textwidth,
	height=\axisdefaultheight,
	xlabel={Tiempo (min)},ylabel={Prom. Abs. (610 nm)},legend pos=north west,legend columns=2,
	legend style={/tikz/every even column/.append style={column sep=0.3cm}}]
	
	\addplot[black,mark=o] table [x=a, y=b, col sep=comma] {2.5.csv};\addlegendentry{Control pH 3.5}
	\addplot[black,mark=triangle*] table [x=a, y=c, col sep=comma] {2.5.csv};\addlegendentry{Pululanasa (0.5U/mL) pH 3.5}	\addplot[black,mark=square*] table [x=a, y=d, col sep=comma] {2.5.csv};\addlegendentry{Control pH 5}
	\addplot[black,mark=diamond*] table[x=a, y=e, col sep=comma] {2.5.csv};\addlegendentry{Pululanasa (0.5U/mL) pH 5}
	\end{axis}
	\end{tikzpicture}
	
	\caption{Gráfica de los datos de muestras reveladas con yodo observados en las Tabla \ref{tab:b4} y \ref{tab:b5}. }
	\label{g2.5}
\end{figure}

\begin{figure}[H]
	\centering
	\begin{tikzpicture}
	\begin{axis}
	[
	xmin=-0.8,
	xmax=30.8,
	width=\textwidth,
	height=\axisdefaultheight,
	xlabel={Tiempo (min)},ylabel={Prom. Abs. (610 nm)},	legend style={at={(0.5,-0.2)},anchor=north},legend columns=2,
	legend style={/tikz/every even column/.append style={column sep=0.3cm}}]
	
	\addplot[black,mark=o] table [x=a, y=b, col sep=comma] {3.1.csv};\addlegendentry{Control isoamilasa 250 U/mL}
	\addplot[black,mark=triangle] table [x=a, y=c, col sep=comma] {3.1.csv};\addlegendentry{Isoamilasa 250 U/mL}	\addplot[black,mark=square*] table [x=a, y=d, col sep=comma] {3.1.csv};\addlegendentry{Control isoamilasa 2,500 U/mL}
	\addplot[black,mark=diamond*] table[x=a, y=e, col sep=comma] {3.1.csv};\addlegendentry{Isoamilasa 2,500 U/mL}
	\end{axis}
	\end{tikzpicture}
	
	\caption{Gráfica de los datos observados en la Tabla \ref{tab:c1}.}
	\label{g3.1}
\end{figure}

\begin{figure}[H]
	\begin{tikzpicture}
\begin{axis}
[%xtick={0,5,10,15,20},
xmin=-1,
xmax=61,
width=\textwidth,
height=\axisdefaultheight,
xlabel={Tiempo (min)},ylabel={Prom. Abs. (610 nm)},legend style={at={(0.5,-0.2)},anchor=north},%legend style={text width=1.6cm, minimum height=1cm},
 legend columns=2]

\addplot[black,mark=triangle*] table[x=a, y=b, col sep=comma] {3.2.csv};\addlegendentry{Control temp. 30ºC }
\addplot[black,mark=o] table [x=a, y=c, col sep=comma] {3.2.csv};\addlegendentry{Isoamilasa (2,500 U/mL) temp. 30ºC}
\addplot[black,mark=x] table[x=a, y=d, col sep=comma] {3.2.csv};\addlegendentry{Control temp. 35ºC}
\addplot[black,mark=triangle ] table[x=a, y=e, col sep=comma] {3.2.csv};\addlegendentry{Isoamilasa (2,500 U/mL) temp. 35ºC}
\addplot[black,mark=diamond*] table[x=a, y=f, col sep=comma] {3.2.csv};\addlegendentry{Control temp. 40ºC}
\addplot[black,mark=star] table[x=a, y=g, col sep=comma] {3.2.csv};\addlegendentry{Isoamilasa (2,500 U/mL) temp. 40ºC}
\addplot[black,mark=square*] table[x=a, y=h, col sep=comma] {3.2.csv};\addlegendentry{Control temp. 60ºC}
\addplot[black,mark=square] table[x=a, y=i, col sep=comma] {3.2.csv};\addlegendentry{Isoamilasa (2,500 U/mL) temp. 60ºC}
\end{axis}
\end{tikzpicture}
	\caption{Gráfica de los datos observados en las Tabla \ref{tab:c2} y \ref{tab:c3}. }
	\label{3.2}
\end{figure}

\begin{figure}[H]
	\centering
	\begin{tikzpicture}
	\begin{axis}
	[
	xmin=-0.8,
	xmax=50.8,
	width=\textwidth,
	height=\axisdefaultheight,
	xlabel={Tiempo (min)},ylabel={Prom. Abs. (610 nm)},legend pos=north west,legend columns=1,
	legend style={/tikz/every even column/.append style={column sep=0.3cm}}]
	
	
	\addplot[black,mark=o] table [x=a, y=b, col sep=comma] {3.3.csv};\addlegendentry{Glucosa 0.0 mg/mL}
	\addplot[black,mark=triangle*] table [x=a, y=c, col sep=comma] {3.3.csv};\addlegendentry{Glucosa 0.1 mg/mL}
	\addplot[black,mark=square*] table [x=a, y=d, col sep=comma] {3.3.csv};\addlegendentry{Glucosa 0.2 mg/mL}
	\end{axis}
	\end{tikzpicture}
	
	\caption{Gráfica de los datos de muestras reveladas con yodo, observados en las Tablas \ref{tab:c3} y  \ref{tab:c4}. }
	\label{g3.3}
\end{figure}


\begin{figure}[H]
	\centering
	\begin{tikzpicture}
	\begin{axis}
	[
	xmin=-0.8,
	xmax=40.8,
	width=\textwidth,
	height=\axisdefaultheight,
	xlabel={Tiempo (min)},ylabel={Prom. Abs. (550 nm)},legend style={at={(0.5,-0.2)},anchor=north},legend columns=2,
	legend style={/tikz/every even column/.append style={column sep=0.3cm}}]
	
	
	\addplot[black,mark=o] table [x=a, y=b, col sep=comma] {4.1.nuevo.csv};\addlegendentry{Control exp. 1}
	\addplot[black,mark=triangle] table [x=a, y=c, col sep=comma] {4.1.nuevo.csv};\addlegendentry{Pululanasa (0.5U/mL) exp. 1}
	\addplot[black,mark=diamond] table [x=a, y=d, col sep=comma] {4.1.nuevo.csv};\addlegendentry{Control exp. 2}
	\addplot[black,mark=square*] table [x=a, y=e, col sep=comma] {4.1.nuevo.csv};\addlegendentry{Pululanasa (0.5U/mL) exp. 2}
	\addplot[black,mark=square] table [x=a, y=f, col sep=comma] {4.1.nuevo.csv};\addlegendentry{Control exp. 3}
	\addplot[black,mark=asterisk] table [x=a, y=g, col sep=comma] {4.1.nuevo.csv};\addlegendentry{Pululanasa (0.5U/mL) exp. 3}
	\end{axis}
	\end{tikzpicture}
	
	\caption{Gráfica de los datos  obtenidos de muestras reveladas con DNS, observados en las Tablas  \ref{tab:e1.1},  \ref{tab:e2.1} y  \ref{tab:e3.1}. }
	\label{g4.1}
\end{figure}


\begin{figure}[H]
	\centering
	\begin{tikzpicture}
	\begin{axis}
	[
	xmin=-0.8,
	xmax=40.8,
	width=\textwidth,
	height=\axisdefaultheight,
	xlabel={Tiempo (min)},ylabel={Prom. Abs. (610 nm)},legend style={at={(0.5,-0.2)},anchor=north},legend columns=2,
	legend style={/tikz/every even column/.append style={column sep=0.3cm}}]
	
	
	\addplot[black,mark=o] table [x=a, y=b, col sep=comma] {4.2.csv};\addlegendentry{Control exp. 1}
	\addplot[black,mark=triangle] table [x=a, y=c, col sep=comma] {4.2.csv};\addlegendentry{Pululanasa (0.5U/mL) exp. 1}
	\addplot[black,mark=diamond] table [x=a, y=d, col sep=comma] {4.2.csv};\addlegendentry{Control exp. 2}
	\addplot[black,mark=square*] table [x=a, y=e, col sep=comma] {4.2.csv};\addlegendentry{Pululanasa (0.5U/mL) exp. 2}
	\addplot[black,mark=square] table [x=a, y=f, col sep=comma] {4.2.csv};\addlegendentry{Control exp. 3}
	\addplot[black,mark=asterisk] table [x=a, y=g, col sep=comma] {4.2.csv};\addlegendentry{Pululanasa (0.5U/mL) exp. 3}
	\end{axis}
	\end{tikzpicture}
	
	\caption{Gráfica de los datos  obtenidos de muestras reveladas con yodo, observados en las Tablas  \ref{tab:e1.1},  \ref{tab:e2.1} y  \ref{tab:e3.1}. }
	\label{g4.2}
\end{figure}


\begin{figure}[H]
	\centering
	\begin{tikzpicture}
	\begin{axis}
	[
	xmin=-0.8,
	xmax=40.8,
	width=\textwidth,
	height=\axisdefaultheight,
	xlabel={Tiempo (min)},ylabel={Prom. Abs. (550 nm)},legend style={at={(0.5,-0.2)},anchor=north},legend columns=2,
	legend style={/tikz/every even column/.append style={column sep=0.3cm}}]
	
	
	\addplot[black,mark=o] table [x=a, y=b, col sep=comma] {4.3.csv};\addlegendentry{Control exp. 1}
	\addplot[black,mark=triangle] table [x=a, y=c, col sep=comma] {4.3.csv};\addlegendentry{Isoamilasa (2,500 U/mL) exp. 1}
	\addplot[black,mark=diamond] table [x=a, y=d, col sep=comma] {4.3.csv};\addlegendentry{Control exp. 2}
	\addplot[black,mark=square*] table [x=a, y=e, col sep=comma] {4.3.csv};\addlegendentry{Isoamilasa (2,500 U/mL) exp. 2}
	\addplot[black,mark=square] table [x=a, y=f, col sep=comma] {4.3.csv};\addlegendentry{Control exp. 3}
	\addplot[black,mark=asterisk] table [x=a, y=g, col sep=comma] {4.3.csv};\addlegendentry{Isoamilasa (2,500 U/mL) exp. 3}
	\end{axis}
	\end{tikzpicture}
	
	\caption{Gráfica de los datos  obtenidos de muestras reveladas con DNS, observados en las Tablas  \ref{tab:e1.2},  \ref{tab:e2.2} y  \ref{tab:e3.2}. }
	\label{g4.3}
\end{figure}

\begin{figure}[H]
	\centering
	\begin{tikzpicture}
	\begin{axis}
	[
	xmin=-0.8,
	xmax=40.8,
	width=\textwidth,
	height=\axisdefaultheight,
	xlabel={Tiempo (min)},ylabel={Prom. Abs. (610 nm)},legend style={at={(0.5,-0.2)},anchor=north},legend columns=2,
	legend style={/tikz/every even column/.append style={column sep=0.3cm}}]
	
	
	\addplot[black,mark=o] table [x=a, y=b, col sep=comma] {4.4.csv};\addlegendentry{Control exp. 1}
	\addplot[black,mark=triangle] table [x=a, y=c, col sep=comma] {4.4.csv};\addlegendentry{Isoamilasa (2,500 U/mL) exp. 1}
	\addplot[black,mark=diamond] table [x=a, y=d, col sep=comma] {4.4.csv};\addlegendentry{Control exp. 2}
	\addplot[black,mark=square*] table [x=a, y=e, col sep=comma] {4.4.csv};\addlegendentry{Isoamilasa (2,500 U/mL) exp. 2}
	\addplot[black,mark=square] table [x=a, y=f, col sep=comma] {4.4.csv};\addlegendentry{Control exp. 3}
	\addplot[black,mark=asterisk] table [x=a, y=g, col sep=comma] {4.4.csv};\addlegendentry{Isoamilasa (2,500 U/mL) exp. 3}
	\end{axis}
	\end{tikzpicture}
	
	\caption{Gráfica de los datos  obtenidos de muestras reveladas con yodo, observados en las Tablas  \ref{tab:e1.2},  \ref{tab:e2.2} y  \ref{tab:e3.2}. }
	\label{g4.4}
\end{figure}

\begin{figure}[H]
	\centering
	\begin{tikzpicture}
	\begin{axis}
	[
	xmin=-0.8,
	xmax=40.8,
	width=\textwidth,
	height=\axisdefaultheight,
	xlabel={Tiempo (min)},ylabel={Prom. Abs. (550 nm)},legend style={at={(0.5,-0.2)},anchor=north},legend columns=2,
	legend style={/tikz/every even column/.append style={column sep=0.3cm}}]
	
	
	\addplot[black,mark=o] table [x=a, y=b, col sep=comma] {4.5.csv};\addlegendentry{Control exp. 1}
	\addplot[black,mark=triangle] table [x=a, y=c, col sep=comma] {4.5.csv};\addlegendentry{Sémola modificada exp. 1}
	\addplot[black,mark=diamond] table [x=a, y=d, col sep=comma] {4.5.csv};\addlegendentry{Control exp. 2}
	\addplot[black,mark=square*] table [x=a, y=e, col sep=comma] {4.5.csv};\addlegendentry{Sémola modificada exp. 2}
	\addplot[black,mark=square] table [x=a, y=f, col sep=comma] {4.5.csv};\addlegendentry{Control exp. 3}
	\addplot[black,mark=asterisk] table [x=a, y=g, col sep=comma] {4.5.csv};\addlegendentry{Sémola modificada exp. 3}
	\end{axis}
	\end{tikzpicture}
	
	\caption{Gráfica de los datos  obtenidos de muestras reveladas con DNS, observados en las Tablas  \ref{tab:e1.3},  \ref{tab:e2.3} y  \ref{tab:e3.3}. }
	\label{g4.5}
\end{figure}


\begin{figure}[H]
	\centering
	\begin{tikzpicture}
	\begin{axis}
	[
	xmin=-0.8,
	xmax=40.8,
	width=\textwidth,
	height=\axisdefaultheight,
	xlabel={Tiempo (min)},ylabel={Prom. Abs. (610 nm)},legend style={at={(0.5,-0.2)},anchor=north},legend columns=2,
	legend style={/tikz/every even column/.append style={column sep=0.3cm}}]
	
	
	\addplot[black,mark=o] table [x=a, y=b, col sep=comma] {4.6.csv};\addlegendentry{Control exp. 1}
	\addplot[black,mark=triangle] table [x=a, y=c, col sep=comma] {4.6.csv};\addlegendentry{Sémola modificada exp. 1}
	\addplot[black,mark=diamond] table [x=a, y=d, col sep=comma] {4.6.csv};\addlegendentry{Control exp. 2}
	\addplot[black,mark=square*] table [x=a, y=e, col sep=comma] {4.6.csv};\addlegendentry{Sémola modificada exp. 2}
	\addplot[black,mark=square] table [x=a, y=f, col sep=comma] {4.6.csv};\addlegendentry{Control exp. 3}
	\addplot[black,mark=asterisk] table [x=a, y=g, col sep=comma] {4.6.csv};\addlegendentry{Sémola modificada exp. 3}
	\end{axis}
	\end{tikzpicture}
	
	\caption{Gráfica de los datos  obtenidos de muestras reveladas con yodo, observados en las Tablas  \ref{tab:e1.3},  \ref{tab:e2.3} y  \ref{tab:e3.3}. }
	\label{g4.6}
\end{figure}



\section{Conclusión}
\section{Bibliografía}



	


\end{document}